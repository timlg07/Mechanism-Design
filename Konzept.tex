\documentclass[a4paper,11pt]{article}
\usepackage[T1]{fontenc}
\usepackage[utf8]{inputenc}
\usepackage[ngerman]{babel}
\usepackage{amssymb}
\usepackage{euler}
\usepackage{amsmath,amsthm}
\usepackage{enumerate}
\usepackage[inner=1.5in,outer=1in]{geometry}
\usepackage{todonotes}
\usepackage[ngerman,onelanguage,linesnumbered]{algorithm2e}
\usepackage{fancyhdr}
\usepackage{datetime}
\usepackage[noend]{algpseudocode}
\usepackage{caption,subcaption}
\usepackage{url}

\renewcommand{\bar}[1]{\overline{#1}}

\theoremstyle{definition}
\newtheorem{definition}{Definition}

\theoremstyle{plain}
\newtheorem{proposition}[definition]{Proposition}
\newtheorem{theorem}[definition]{Satz}
\newtheorem{lemma}[definition]{Lemma}
\newtheorem{corollary}[definition]{Korollar}

\theoremstyle{definition}
\newtheorem{bemerkung}[definition]{Bemerkung}
\newtheorem*{bemerkung*}{Bemerkung}

\newtheorem{claim}{Behauptung}
\renewcommand*{\theclaim}{\Alph{claim}}

\parskip=1ex
\parindent=0ex


\title{Einführung in \textit{Mechanism Design} \\ \Large im Seminar ``Algorithmische Spieltheorie''}
\author{Tim Greller}
%\date{Sommersemester 2018}


\begin{document}

\maketitle

\setcounter{page}{0}
\pagenumbering{arabic}
\fancyhead{}
\fancyhead[ER]{\leftmark}
\fancyhead[OL]{\rightmark}
\fancyhead[EL,OR]{\thepage}
\pagestyle{fancy}


- Hauptproblem: umfangreiches Kapitel $\rightarrow$ was weglassen? Aktuell: Teilbereiche, die mir als weniger relevant erschienen sind, weggelassen. Eher Breite abdecken oder bei einzelnen Themen mehr in die Tiefe gehen?

- Aktuell tendenziell zu viel Inhalt, ggf. im Nachhinein kürzen (geht einfacher als neue Themen/ Beweise einfügen).


\section{Einleitung, Definitionen}
\textsmaller{c.a. 4 min; Ähnlich wie bei Kurzvortrag, aber etwas ausführlicher und leichter verständlich.}

- Social Choice: Aggregation von Präferenzen

- Social Welfare Functions, Social Choice Functions

- Strategische Manipulation

- Was ist Mechanism Design?


\section{Arrow's Theorem, Gibbard-Satterthwaite}
\textsmaller{c.a. 11 min; Ähnlich wie bei Kurzvortrag + ausführlicher Beweis zusätzlich.}

- unanimity

- dictators

- IIA (independence of irrelevant alternatives)

- \textbf{Beweis Arrow's Theorem}

- incentive compatible social choice functions: analog zu social welfare functions gilt hier das Gibbard-Satterthwaite Theorem (kein Beweis)


\section{\textit{Mechanism Design} mit Geld}
valuation function: Hinzufügen von Geld als Ausweg um trotz Gibbard-Satterthwaite Theorem incentive compatible social choice functions zu designen.

\subsection{Vickrey's Second Price Auction}
\textsmaller{nochmal kurz wiederholen, etwa 1 min.}

Fazit: trotz privater Informationen und egoistischem Verhalten wird social welfare erreicht.

Ziel: Dies auch im allgemeinen Fall zu erreichen.

\subsection{Vickrey-Clarke-Groves Mechanisms}
\textsmaller{c.a. 4 min}

- VCG Mechanismen maximieren social welfare

- jeder Spieler bekommt die Summe der Werte der anderen Spieler ausbezahlt $\Rightarrow$ Ziel der Spieler: social welfare maximieren durch das Sagen der Wahrheit

- \textbf{Beweis Vickrey-Clarke-Groves Theorem:} Jeder VCG Mechanismus ist incentive compatible

\subsection{Clarke Pivot Rule}
\textsmaller{c.a. 2 min}

- Jeder Spieler zahlt die Differenz zwischen dem social welfare mit und ohne der eigenen Beteiligung.

- VCG Mechanismus mit Clarke Pivot Rule macht keine positiven Transfers

\subsection{Beispiel: Öffentliches Projekt}
\textsmaller{c.a. 3 min}

- öffentliches Bauprojekt mit Kosten $C$ (hierfür eigener Spieler ''Regierung'')

- Projekt wird umgesetzt, wenn Summe der Werte aller Bürger größer als Kosten C

- Alle Bürger werden Betrag 0 zahlen, außer sie wären ausschlaggebend für Erfolg  des Projekts.


% Abschnitte 9.4 - 9.5 werden ausgelassen
\section{Bayesian-Nesh Implementation}
\textsmaller{c.a. 12 min, inkl. Beweis für First Price Auction}

-  A-priori-Wahrscheinlichkeit steht als Grundwissen zur Verfügung. Information statt Annahme des worst-case 

\subsection{First Price Auction}

Spieler A und B bieten um Gegenstand. Beide wollen mehr bieten als der andere, aber weniger als ihr Wert

Gesucht: Strategien, die jeweils die beste Antwort auf die andere Strategie sind (Bayesian equilibrium)

\textbf{Für jeweilige a-priori-Verteilung x ist die Strategie x/2 im Bayesian-Nash equilibrium.}

Für den Verkäufer ist der Umsatz von First- \& Second Price Auction identisch (wenn $w_A$, $w_B$ einheitlich gewählt).


\section{Zusammenfassung und Ausblick}
\textsmaller{c.a. 3 min}

- Kurzes Fazit zu Mechanism Design

- Ausblick, weitere Forschung: z. B. Verwendung von Mechanism Design zur Konfliktlösung und Entscheidungsfindung bei autonomen Fahrzeugen (Lovellette \& Hexmoor 2021: \url{https://doi.org/10.1016/j.iot.2020.100356}) 

\end{document}
